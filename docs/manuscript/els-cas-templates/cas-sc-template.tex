%% 
%% Copyright 2019-2021 Elsevier Ltd
%% 
%% This file is part of the 'CAS Bundle'.
%% --------------------------------------
%% 
%% It may be distributed under the conditions of the LaTeX Project Public
%% License, either version 1.2 of this license or (at your option) any
%% later version.  The latest version of this license is in
%%    http://www.latex-project.org/lppl.txt
%% and version 1.2 or later is part of all distributions of LaTeX
%% version 1999/12/01 or later.
%% 
%% The list of all files belonging to the 'CAS Bundle' is
%% given in the file `manifest.txt'.https://www.overleaf.com/project/62b9aa3c03e7844e405c39e9
%% 
%% Template article for cas-sc documentclass for 
%% single column output.

\documentclass[a4paper,fleqn]{cas-sc}

% If the frontmatter runs over more than one page
% use the longmktitle option.

%\documentclass[a4paper,fleqn,longmktitle]{cas-sc}

%\usepackage[numbers]{natbib}
%\usepackage[authoryear]{natbib}
\usepackage[authoryear,longnamesfirst]{natbib}

%%%Author macros
\def\tsc#1{\csdef{#1}{\textsc{\lowercase{#1}}\xspace}}
\tsc{WGM}
\tsc{QE}
%%%

% Uncomment and use as if needed
%\newtheorem{theorem}{Theorem}
%\newtheorem{lemma}[theorem]{Lemma}
%\newdefinition{rmk}{Remark}
%\newproof{pf}{Proof}
%\newproof{pot}{Proof of Theorem \ref{thm}}

\begin{document}
\let\WriteBookmarks\relax
\def\floatpagepagefraction{1}
\def\textpagefraction{.001}

% Short title
\shorttitle{<short title of the paper for running head>}    

% Short author
\shortauthors{<short author list for running head>}  

% Main title of the paper
\title [mode = title]{<main title>}  

% Title footnote mark
% eg: \tnotemark[1]
\tnotemark[<tnote number>] 

% Title footnote 1.
% eg: \tnotetext[1]{Title footnote text}
\tnotetext[<tnote number>]{<tnote text>} 

% First author
%
% Options: Use if required
% eg: \author[1,3]{Author Name}[type=editor,
%       style=chinese,
%       auid=000,
%       bioid=1,
%       prefix=Sir,
%       orcid=0000-0000-0000-0000,
%       facebook=<facebook id>,
%       twitter=<twitter id>,
%       linkedin=<linkedin id>,
%       gplus=<gplus id>]

\author[1]{Joseph Holler}[orcid=0000-0002-2381-2699]

% Corresponding author indication
\cormark[<corr mark no>]

% Footnote of the first author
\fnmark[<footnote mark no>]

% Email id of the first author
\ead{josephh@middlebury.edu}

% URL of the first author
\ead[url]{<URL>}

% Credit authorship
% eg: \credit{Conceptualization of this study, Methodology, Software}
\credit{<Credit authorship details>}

% Address/affiliation
\affiliation[<aff no>]{organization={},
            addressline={}, 
            city={},
%          citysep={}, % Uncomment if no comma needed between city and postcode
            postcode={}, 
            state={},
            country={}}

\author[1]{Junyi Zhou}

% Footnote of the second author
\fnmark[2]

% Email id of the second author
\ead{zhou@middlebury.edu}

% URL of the second author
\ead[url]{}

% Credit authorship
\credit{}

% Address/affiliation
\affiliation[1]{organization={Middlebury College},
            addressline={}, 
            city={Middlebury},
%          citysep={}, % Uncomment if no comma needed between city and postcode
            postcode={05753}, 
            state={VT},
            country={United States}}

% Corresponding author text
\cortext[1]{Corresponding author}

% Footnote text
\fntext[1]{}

% For a title note without a number/mark
%\nonumnote{}

% Here goes the abstract
\begin{abstract}
Chakraborty (2021) investigates the relationships between COVID-19 rates and demographic characteristics of people with disabilities by county in the lower 48 states.
The study aims to examine public concern that persons with disabilities (PwDs) face disproportionate challenges due to COVID-19.
To investigate this, Chakraborty examines the statistical relationship between confirmed county-level COVID-19 case rates and county-level socio-demographic and disability variables.
Specifically, Chakraborty tests county-level bivariate correlations between COVID-19 incidence against the percentage of disability and socio-demographic category, with a separate hypothesis and model for each subcategory within disability, race, ethnicity, age, and biological sex.
To control for differences between states and geographic clusters of COVID-19 outbreaks, Chakraborty uses five generalized estimating equation (GEE) models to predict the relationship and significance between COVID-19 incidence and disability subgroups within each socio-demographic category while considering inter-county spatial clusters.
Chakraborty (2021) finds significant positive relationships between COVID-19 rates and socially vulnerable demographic categories of race, ethnicity, poverty, age, and biological sex.

This reproduction study is motivated by expanding the potential impact of Chakraborty's study for policy, research, and teaching purposes.
Measuring the relationship between COVID-19 incidence and socio-demographic and disability characteristics can provide important information for public health policy-making and resource allocation.
A fully reproducible study will increase the accessibility, transparency, and potential impact of Chakraborty's (2021) study by publishing a compendium complete with metadata, data, and code.
This will allow other researchers to review, extend, and modify the study and will allow students of geography and spatial epidemiology to learn from the study design and methods.

In this reproduction, we will attempt to identically reproduce all of the results from the original study.
This will include the map of county level distribution of COVID-19 incidence rates (Fig. 1), the summary statistics for disability and sociodemographic variables and bivariate correlations with county-level COVID-19 incidence rate (Table 1), and the GEE models for predicting COVID-19 county-level incidence rate (Table 2).
A successful reproduction should be able to generate identical results as published by Chakraborty (2021).

The reproduction study data and code will be made available in a GitHub repository to the greatest extent that licensing and file sizes permit.
The repository will be made public at [github.com/HEGSRR/RPr-Chakraborty2021](https://github.com/HEGSRR/RPr-Chakraborty2021).
To the greatest extent possible, the reproduction will be implemented with R markdown using packages geepack for the generalized estimating equation and SpatialEpi for the spatial scan statistics.
\end{abstract}

% Use if graphical abstract is present
%\begin{graphicalabstract}
%\includegraphics{}
%\end{graphicalabstract}

% Research highlights
\begin{highlights}
\item 
\item 
\item 
\end{highlights}

% Keywords
% Each keyword is seperated by \sep
\begin{keywords}
COVID-19 \sep Disability \sep Reproduction Study \sep Kulldorff Spatial Scan Statistic \sep United States \sep Generalized Estimating Equations
\end{keywords}

\maketitle

% Main text
\section{}\label{}

% Numbered list
% Use the style of numbering in square brackets.
% If nothing is used, default style will be taken.
%\begin{enumerate}[a)]
%\item 
%\item 
%\item 
%\end{enumerate}  

% Unnumbered list
%\begin{itemize}
%\item 
%\item 
%\item 
%\end{itemize}  

% Description list
%\begin{description}
%\item[]
%\item[] 
%\item[] 
%\end{description}  

% Figure
\begin{figure}[<options>]
	\centering
		\includegraphics[<options>]{}
	  \caption{}\label{fig1}
\end{figure}


\begin{table}[<options>]
\caption{}\label{tbl1}
\begin{tabular*}{\tblwidth}{@{}LL@{}}
\toprule
  &  \\ % Table header row
\midrule
 & \\
 & \\
 & \\
 & \\
\bottomrule
\end{tabular*}
\end{table}

% Uncomment and use as the case may be
%\begin{theorem} 
%\end{theorem}

% Uncomment and use as the case may be
%\begin{lemma} 
%\end{lemma}

%% The Appendices part is started with the command \appendix;
%% appendix sections are then done as normal sections
%% \appendix

\section{}\label{}

% To print the credit authorship contribution details
\printcredits

%% Loading bibliography style file
%\bibliographystyle{model1-num-names}
\bibliographystyle{cas-model2-names}

% Loading bibliography database
\bibliography{}

% Biography
\bio{}
% Here goes the biography details.
\endbio

\bio{pic1}
% Here goes the biography details.
\endbio

\end{document}

